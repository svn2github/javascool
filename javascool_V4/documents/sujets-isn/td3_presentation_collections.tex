%\documentclass[serif,mathserif,8pt,handout]{beamer}
\documentclass[serif,mathserif,10pt,handout]{beamer}
\usepackage{pxfonts}
\usepackage{eulervm}
\usepackage{alltt}
\usepackage{color,latexsym,calc,array,amssymb,listings} 
\usetheme{m2r}
\usepackage[latin1]{inputenc}
\usepackage{ulem}

\definecolor{OliveGreen}    {cmyk}{0.64,0,0.95,0.40}

\def\lstlanguagefiles{lstjava.sty}
\lstloadlanguages{Java}
\lstset{language=Java,commentstyle=\rm\color{red},basicstyle=\ttfamily\small\color{OliveGreen},keywordstyle=\bfseries\color{OliveGreen}}
\lstset{escapeinside={(*@}{@*)}}


\setlength{\parskip}{\baselineskip}


\title[Les collections Java en 10 min]{Les collections Java en 10 minutes} 

\author{David Pichardie}
\date{Jeudi 17 Mars}
\institute{}

\beamerboxesdeclarecolorscheme{darkbox}{blue}{structure!50}
\beamerboxesdeclarecolorscheme{bbox}{structure!75!black}{structure!75!black}

\definecolor{Blue}{cmyk}{1, 0.5, 0, 0.3}
%\newcommand{\notes}{\begin{frame}\frametitle{Notes} \mbox{} \end{frame}}
\newcommand{\notes}{}

%%%%%%%%%%%%%%%%%%%%%%%%%%%%%%%%%%%%%%%%%%%%%%%%

\begin{document}

\frame[plain]{\titlepage}

% \frame<0>[plain]{
%   \begin{center}
% \Large
% \structure{\texttt{List<A>}}\visible<2->{~:~\structure{\texttt{List<String>}}}\visible<3->{, \structure{\texttt{List<Integer>}}}\visible<4->{, \sout{\structure{\texttt{List<int>}}}, ...}

% ~

% ~

% \structure{\texttt{Map<A,B>}}\visible<5->{~:~\structure{\texttt{Map<String,Double>}}}\visible<6->{, \structure{\texttt{Map<String,List<String>>}}}

% \end{center}
% }



\begin{frame}[fragile]{Comment manipuler une liste~~~~\visible<11->{/~~~~Comment manipuler une table}}

\begin{columns}
\column{.5\textwidth}    
\pause
Cr�er une liste vide\\
~~~\lstinline|List<A> liste = new ArrayList<A>();|\\[1ex]
\pause
Lire le $i$�me �l�ment\\
~~~\lstinline|int i = ...; A a = liste.get(i);|\\[1ex]
\pause
Mettre � jour le $i$�me �l�ment\\
~~~\lstinline|int i = ...; A a = ...; liste.set(i,a);|\\[1ex]
\pause
Ajouter un �l�ment en fin de liste\\
~~~\lstinline|A a = ...; liste.add(a);|\\[1ex]
\pause
Ajouter un �l�ment entre la position $i$ et $i+1$\\
~~~\lstinline|int i = ...; A a = ...; liste.add(i,a);|\\[1ex]
\pause
Tester l'appartenance d'un �l�ment\\
~~~\lstinline|A a = ...; boolean test = liste.contains(a);|\\[1ex]
\pause
Obtenir la taille courante\\
~~~\lstinline|int taille = liste.size();|\\[1ex]
\pause
Afficher une liste\\
~~~\lstinline|echo(liste);|\\[1ex]
\pause
It�rer sur une liste\\
~~~\lstinline|for (A x : liste)|\\
~~~~~~\lstinline|{...traitement sur x...}|



\column{.5\textwidth}    
\pause
\pause
Cr�er une table vide\\
~~~\lstinline|Map<A,B> table = new HashMap<A,B>();|\\[1ex]
\pause
Tester l'existence d'une entr�e\\
~~~\lstinline|A a = ...; boolean test = table.containsKey(a);|\\[1ex]
\pause
Lire une entr�e\\
~~~\lstinline|A a = ...; B b = table.get(a);|\\[1ex]
\pause
Mettre � jour une entr�e (ou la cr�er)\\
~~~\lstinline|A a = ...; B b = ...; table.put(a,b);|\\[1ex]
\pause
Obtenir la taille courante\\
~~~\lstinline|int taille = table.size();|\\[1ex]
\pause
It�rer sur une table\\
~~~\lstinline|for (A x : table.keySet())|\\
~~~~~~\lstinline|{...traitement sur x et table.get(x)...}|
\pause
Afficher une table\\
~~~\lstinline|echo(table);|\\[1ex]
\pause
Obtenir l'ensemble des cl�s\\
~~~\lstinline|Set<A> dom = table.keySet()|\\

~

Remarque :  \lstinline|Set<A>| $\simeq$ \lstinline|List<A>| sans r�p�tition
\end{columns}


\end{frame}
\end{document}

\frame[plain,fragile]{
  \begin{lstlisting}
    Map<A,B>
  \end{lstlisting}
}

\end{document}
