\documentclass[a4paper,11pt]{article}

%% \LoadClass[a4paper,11pt]{article}

\usepackage{a4wide,graphicx}%

%\usepackage[francais]{babel}%
\usepackage[utf8]{inputenc}%
\usepackage[T1]{fontenc}%

\usepackage{url} \urlstyle{sf}%

\usepackage{mathpazo}%
\let\bfseriesaux=\bfseries%
\renewcommand{\bfseries}{\sffamily\bfseriesaux}

\newcounter{probleme} \newcounter{question}[probleme]
\setcounter{probleme}{0} \def\theprobleme{\Roman{probleme}}
\def\thequestion{\theprobleme.\arabic{question}}

\newenvironment{probleme}[1]%
{\refstepcounter{probleme}
\subsection*{Activité \theprobleme. #1}}%
{}

\newenvironment{question}%
{\refstepcounter{question} \description%
\item[Exercice \thequestion.] \bgroup}%
{\egroup\enddescription}

\newenvironment{remarque}%
{\description\item[Remarque.]\sl}%
{\enddescription}

\newenvironment{exemple}%
{\description\item[Exemple.]\sl}%
{\enddescription}

\newenvironment{programme}%
{\verbatim}{\endverbatim}%

\usepackage{listings}%

\lstset{%
  basicstyle=\sffamily,%
  columns=fullflexible,%
   language=java,%
   frame=lb,%
   frameround=fftf,%
}%

\def\|{\lstinline|}%

\lstnewenvironment{lstlistingtt}%
{\lstset{%
    basicstyle=\tt,%
    columns=flexible,%
  }}%
{}

\lstnewenvironment{lstlistingsh}%
{\lstset{%
  language=sh,%
  }}%
{}

%% \lstMakeShortInline{|}

\parskip=0.3\baselineskip%
\sloppy%

\begin{document}

\title{Enseignement de spécialité\\
Informatique et sciences du numérique\\
Formation des IA-IPR et chargés de mission\\
Atelier de programmation 4}

\date{Jeudi 17 mars}

\author{David Pichardie, Luc Bougé}
\maketitle

Dans cette activité, nous allons manipuler des grands textes sous
forme de listes de mots. Nous utiliserons des structures de données
plus avancées que les tableaux~: les collections qui vous seront
présentées en séance.
  
La fonction 
\begin{lstlisting}
List<String> lireGrandTexte(String nomDefichier)
\end{lstlisting}
vous permet de
charger un texte sous la forme d'une liste de chaînes de
caractères. Nous travaillerons plus spécialement sur deux textes:
\emph{La Marseillaise} et \emph{Les Misérables}. Le premier est obtenu
par la commande
\begin{lstlisting}
List<String> la_marseillaise = lireGrandTexte("La_Marseillaise.txt")  ;
\end{lstlisting}
et le deuxième par 
\begin{lstlisting}
List<String> les_miserables = lireGrandTexte("Les_Miserables.txt");
\end{lstlisting}
Ces deux textes sont libres de droit et consultable en ligne. Pour
simplifier, nous enlevons toutes les majuscules et supprimons la ponctuation
de ces textes.

\begin{question}
  Écrire un programme 
  \begin{lstlisting}
    void afficheLaMarseillaise()
  \end{lstlisting}
qui charge le texte de la \emph{Marseillaise} et affiche tous les mots qui le
  compose a l'aide d'une itération. 
\end{question}

\begin{remarque}
  Attention, ne tenter pas la même opération avec \emph{Les
    Misérables} car le document est beaucoup trop gros pour les
  capacités d'affichage de Javascool (plus de trois millions de
  caractères)~!
\end{remarque}

\begin{question}
  Écrire un programme 
  \begin{lstlisting}
    int nombreDeMots(List<String> liste)
  \end{lstlisting}
  qui compte le nombre de mots dans une liste de mots. Tester sur le
  texte des \emph{Misérables}.  Remarquer que cette
  fonctionnalité existe déjà avec l'opération \|l.size();| sur une
  liste \|l| mais nous vous demandons de réaliser vous-même ce calcul
  à l'aide d'une itération.
\end{question}

\begin{question}
  La liste calculée par la fonction \|lireGrandTexte| contient une chaîne
  de caractères spéciale "$<$NOUVELLE PAGE$>$" qui est insérée dans la
  liste à chaque changement de page. Utiliser ce mot pour calculer le
  nombre de page avec une fonction
\begin{lstlisting} 
int nombreDePages(List<String> liste)  
\end{lstlisting}
Compter ainsi le nombre de pages dans \emph{Les Misérables}.
\end{question}


\begin{question}
  Écrire un programme 
  \begin{lstlisting}
    Map<String,Integer> tableDesFrequences(List<String> liste)
  \end{lstlisting}
  qui calcule la table d'association reliant chaque mot à sa fréquence
  d'apparition (nombre d'occurrence) dans la liste.
Tester sur le
  texte de  \emph{La Marseillaise}.
\end{question}

\begin{question}
  Écrire un programme 
  \begin{lstlisting}
    int nombreDeMotsDistincts(List<String> liste)
  \end{lstlisting}
  qui compte le nombre de mots distincts dans une liste de mots. Tester sur le
  texte des \emph{Misérables}.  
\end{question}

\begin{question}
  Écrire un programme 
  \begin{lstlisting}
    void portraitDeCosette()
  \end{lstlisting}
  qui affiche  les 100 premiers mots rencontrés après la première apparition du mot "cosette" (sans majuscules).
\end{question}

\begin{question}
Nous vous fournissons la fonction
\begin{lstlisting}
  Map<String,Integer> triMapParValeursDecroissantes(Map<String,Integer> table)
\end{lstlisting}
qui renvoie une version tri\'ee d'un table en triant les fr\'equences par ordre croissant. Utiliser cette fonction pour \'ecrire un programme
  \begin{lstlisting}
    void lesPlusFrequents(List<String> liste, int N)
  \end{lstlisting}
  qui affiche les \|N| mots les plus fréquents de la liste \|liste| en ignorant les mots de moins de 4 lettres.
  Utiliser cette opération avec \|N| égal à 30 pour voir citer les personnages principaux des \emph{Misérables}.
\end{question}

\begin{question}
Utiliser le mot spécial
"$<$NOUVELLE PAGE$>$" pour calculer une table d'index pour chaque mot d'au moins 4 lettres 
du texte avec une fonction
\begin{lstlisting} 
Map<String,List<String>> tableIndex(List<String> liste)  
\end{lstlisting}
Afficher la table d'index des 30 mots les plus fréquents.
\end{question}



%\noindent\textbf{Remerciements :} Les textes proposés dans ce sujet sont libres de droit est disponibles à l'adresse suivante \url{http://www.gutenberg.org}.
\end{document}
