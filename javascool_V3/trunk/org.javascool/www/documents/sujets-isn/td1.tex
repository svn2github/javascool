\documentclass[a4paper,11pt]{article}

%% \LoadClass[a4paper,11pt]{article}

\usepackage{a4wide}%

\usepackage[francais]{babel}%
\usepackage[utf8]{inputenc}%
\usepackage[T1]{fontenc}%

\usepackage{url} \urlstyle{sf}%

\usepackage{mathpazo}%
\let\bfseriesaux=\bfseries%
\renewcommand{\bfseries}{\sffamily\bfseriesaux}

\newcounter{activite}%
\newcounter{question}[activite]%
\setcounter{activite}{0}% 
\def\theactivite{\Roman{activite}}%
\def\thequestion{\theactivite.\arabic{question}}%

\newenvironment{activite}[1]%
{\refstepcounter{activite}
\subsection*{Activité \theactivite. #1}}%
{}

\newenvironment{question}%
{\refstepcounter{question} \description%
\item[Exercice \thequestion.] \bgroup}%
{\egroup\enddescription}

\newenvironment{remarque}%
{\description\item[Remarque.]\sl}%
{\enddescription}

\newenvironment{exemple}%
{\description\item[Exemple.]\sl}%
{\enddescription}

\newenvironment{programme}%
{\verbatim}{\endverbatim}%

\usepackage{listings}%

\lstset{%
  basicstyle=\sffamily,%
  columns=fullflexible,%
   language=java,%
   frame=lb,%
   frameround=fftf,%
}%

\def\|{\lstinline|}%

\lstnewenvironment{lstlistingtt}%
{\lstset{%
    basicstyle=\tt,%
    columns=flexible,%
  }}%
{}

\lstnewenvironment{lstlistingsh}%
{\lstset{%
  language=sh,%
  }}%
{}

%% \lstMakeShortInline{|}

\parskip=0.3\baselineskip%
\sloppy%

\begin{document}

\title{Enseignement de spécialité\\
Informatique et sciences du numérique\\
Formation des IA-IPR et chargés de mission\\
Atelier de programmation 2}

\date{Mardi 15 mars}

\author{David Pichardie, Luc Bougé}

\maketitle

Nous utiliserons pour cet atelier l'environnement Javascool. Dans un
premier temps, il convient de vous familiariser avec cet
environnement.
\begin{enumerate}
\item Télécharger et lancer l'application Javascool
  (\url{http://javascool.gforge.inria.fr/}, onglet \emph{lancement}).
\item Choisir l'activité \emph{Découvrir les ingrédients des algorithmes}.
\item Réaliser les exercices qui suivent, ou bien commencer par suivre le parcours d'initiation proposé par Javascool.
\end{enumerate}
Ce premier atelier vous propose 3~niveaux d'activités de difficulté croissante. Choisissez un niveau
en fonction de votre familiarité avec la programmation.

\begin{activite}{Premiers pas en programmation}

  Nous allons programmer un système expert rudimentaire qui dialoguera
  avec l'utilisateur et adaptera son discours aux réponses de
  l'utilisateur. Voici un exemple de conversation entre le système et
  l'utilisateur. Dans l'exemple suivant, chaque entrée de
  l'utilisateur est précédée du texte \verb|[Entrée :]|.
\begin{verbatim}
Bonjour ! Quel est votre nom ?
[Entrée :] Haddock
C'est noté. Et votre prénom ?
[Entrée :] Archibald
Êtes-vous un homme ou une femme ?
[Entrée :] homme
Désolé, je n'ai pas compris votre réponse. Êtes-vous un homme ou une femme ?
[Entrée :] un homme
Parfait. Je suis enchanté de faire votre connaissance M. Archibald Haddock.

Est-ce indiscret de vous demander votre âge ?
[Entrée :] non
Dans ce cas, quel âge avez-vous M. Haddock ?
[Entrée :] 45
Félicitations vous êtes dans la force de l'âge !

Savez-vous que je suis excellent en calcul mental ? 
Proposez-moi un grand nombre s'il vous plaît.
[Entrée:] 3
Ah, c'est bien petit 3 ! Ne pouvez-vous pas me proposer 
un nombre un peu plus grand ?
[Entrée:] 337
C'est mieux. 
Je le note. Proposez-moi un autre grand nombre s'il vous plaît.
[Entrée:] 3456
C'est noté. Et bien... 337 fois 3456 font 1164672 ! 
Je vous sens impressionné...

Je suis aussi capable de compter très vite ! 
Jusqu'à combien voulez-vous que je compte ? 
[Entrée:] 14
1
2
3
4
5
6
7
8
9
10
11
12
13
14

Merci pour ce bon moment. A bientôt Archibald.
\end{verbatim}

\begin{question}
  Proposez un programme Javascool permettant d'obtenir ce type de
  conversation en tenant compte des différentes réponses de
  l'utilisateur (sexe, âge, etc.), et en validant le format des
  réponses données. Par exemple, lorsqu'on attend une réponse oui ou
  non, toute autre réponse doit être refusée.
\end{question}

\begin{remarque}
  Attention! Pour tester l'égalité de deux chaînes, il ne faut pas
  utiliser l'opération \|==|. En effet, celle-ci teste
  \emph{l'identité} des deux chaînes et non le fait qu'elles soient
  formées des mêmes caractères. Il faut utiliser une fonction appelée
  \|equal|.
\begin{lstlisting}
String s = readString();
if (equal(s, "Haddock")) ...
\end{lstlisting}
\end{remarque}

\begin{question}
  Pour le produit de deux nombres, ne peut-on pas pousser le système à
  la faute en lui proposant des nombres trop grands à multiplier?
  Comment filtrer les entrées de l'utilisateur pour remédier à ce
  problème?
\end{question}

Pour éviter une saisie fastidieuse des phrases précédentes, vous
pouvez copier-coller les phrases placées dans la version Javascool de
cet énoncé.
\end{activite}

\begin{activite}{Premiers pas en programmation des tableaux}

Nous nous intéressons maintenant à la manipulation des anagrammes de
mots. Étant donné une collection de mots donnée sous forme d'un
tableaux de chaînes de caractères, nous laissons l'utilisateur
proposer un mot et nous l'informons des différents anagrammes de ce
mot qui sont présents dans la collection.


\begin{exemple}
Supposons que le tableau initial soit le suivant\footnote{Il vous
    sera nécessaire de recopier cette déclaration de tableau pour faire
    l'exercice. Un copier-coller à partir de l'énoncé fourni dans Javascool
    sera certainement la façon la plus rapide de le faire.}:
\begin{lstlisting}
String[] choixAnagrammes = 
  { "marion", "aimer", "badinage", "niche", "aspirine", "chine",
     "aspirine", "aube", "baignade", "chien", "manoir", "beau", 
     "romain", "imaginer", "parisien", "migraine", "marie"}
\end{lstlisting}
L'utilisateur est invité à saisir un mot.
\begin{verbatim}
Proposez-moi un mot !
[Entrée :] beau
aube est un anagramme de beau
beau est un anagramme de beau
nombre d'anagrammes : 2 
\end{verbatim}
\end{exemple}

Afin de réaliser un tel programme, nous allons devoir tester si un mot
est un anagramme d'un autre.  Pour ce faire, nous vous proposons de
manipuler les chaînes de caractères comme des tableaux de
caractères. La fonction
\begin{lstlisting}
char[] chaineVersLettres(String s)
\end{lstlisting}
vous permet d'obtenir le tableau de caractères correspondant à une
chaîne de caractère.  Le type \|char| est un type Java primitif
qui représente les caractères. Contrairement aux chaînes, on peut
utiliser les opérateurs de comparaisons arithmétiques usuels pour
comparer deux caractères selon leur position dans l'alphabet.

Nous vous proposons une deuxième fonction
\begin{lstlisting}
String lettresVersChaine(char[] t)
\end{lstlisting}
qui permet de reformer une chaîne à partir d'un tableau de caractères
pour pouvoir l'imprimer.

Pour tester si un mot est un anagramme d'un autre nous allons
\emph{trier} chacun des deux mots en utilisant l'ordre alphabétique
puis vérifier si nous obtenons la même séquence de lettres.  Par
exemple, si nous trions \|"beau"| et \|"aube"| nous
obtenons dans les deux cas \|"abeu"| ce qui nous assure que ce
sont des anagrammes.

Dans un premier temps, nous vous proposons d'utiliser les fonctions de
la bibliothèque standard de Java pour faire ce tri. On peut par
exemple utiliser:
\begin{lstlisting}
void java.utils.Arrays.sort(char[] t)
\end{lstlisting}

\begin{exemple}
Le programme suivant affiche \|"abeu"|.
  \begin{lstlisting}
char[] t = chaineVersLettres("beau");
java.util.Arrays.sort(t);
echo(lettresVersChaine(t));
  \end{lstlisting}
\end{exemple}

\begin{question}
  Écrire une fonction
\begin{lstlisting}
String normalise(String s)
\end{lstlisting}
qui prend en entrée une chaîne \|s| et renvoie une nouvelle
chaîne obtenue en triant par ordre alphabétique les lettres de
\|s|.
\end{question}

\begin{question}
  Écrire une fonction
\begin{lstlisting}
boolean testAnagramme(String s1, String s2)
\end{lstlisting}
  qui teste si deux chaîne \|s1| et \|s2| sont anagrammes
  l'une de l'autre grâce à un tri.
\end{question}

\begin{question}
  Proposer un programme pour interagir avec l'utilisateur selon le
  scénario proposé plus haut.
\end{question}

Nous allons maintenant tenter d'écrire notre propre fonction de tri.
\begin{lstlisting}
void triSelection(char[] t)
\end{lstlisting}

\begin{question}
  Écrire tout d'abord une fonction
\begin{lstlisting}
void permute(char[] t, int i, int j)
\end{lstlisting}
  qui échange les caractères à la position \|i| et \|j|
  dans le tableau \|t|. On suppose que \|i| et \|j|
  sont des indices valides dans \|t|.
\end{question}

\begin{question}
  Écrire une fonction
\begin{lstlisting}
int indexMin(char[] t, int i)
\end{lstlisting}
  qui calcule l'indice de la plus petite lettre (pour l'ordre
  alphabétique) de \|t|, entre les indices \|i| et
  \|t.length-1|.
\end{question}

\begin{question}
  Écrire une fonction
\begin{lstlisting}
void triSelection(char[] t) 
\end{lstlisting}
  qui échange les lettres du tableau \|t| pour les mettre dans
  l'ordre alphabétique.
\end{question}

\end{activite}

\begin{activite}{Réapprendre à compter comme à l'école}

  Dans cette activité, nous vous proposons de réapprendre à compter en
  expliquant à l'ordinateur comment poser et résoudre les opérations
  arithmétiques élémentaires: addition, multiplication, soustraction,
  division. Nous allons pour cela manipuler les nombres comme des
  tableaux de chiffres décimaux. Dans un deuxième temps nous vous
  proposons d'adapter ces programmes à l'arithmétique binaire en
  manipulant cette fois des tableaux de bits.

  Nous manipulerons pour cette activité le type \|Chiffre| suivant.
\begin{lstlisting}
  enum Chiffre { ZERO, UN, DEUX, TROIS, QUATRE, CINQ, SIX, SEPT, HUIT, NEUF }
\end{lstlisting}
Un nombre sera manipulé sous la forme d'un tableau de valeurs de type
\|Chiffre|.

Pour résoudre les opérations arithmétiques élémentaires, il faut
connaître ses tables d'addition et de multiplication.  Ces tables vous
sont fournis sous la forme du type \|NombreA2Chiffres| qui
encapsule un chiffre des dizaines et un chiffre des unités. Ce type
abstrait est munis des opérations suivantes:
\begin{itemize}

\item \|Chiffre unite(NombreA2Chiffres n)| renvoie le chiffe des
  unités d'un nombre \|n|.

\item \|Chiffre dizaine(NombreA2Chiffres n)| renvoie le chiffe des
  dizaines d'un nombre \|n|.

\item \|NombreA2Chiffres nb(Chiffre dizaine, Chiffre unite)| fabrique
  un nombre à partir d'un chiffre \|dizaine| des dizaines et \|unite|
  des unités.

\item \|NombreA2Chiffres plus(NombreA2Chiffres x, NombreA2Chiffres y)|
  renvoie le nombre obtenu en faisant la somme des deux nombres \|x|
  et \|y|.

\item \|NombreA2Chiffres moins(NombreA2Chiffres x, NombreA2Chiffres y)| 
  renvoie le nombre obtenu en faisant la différence des deux
  nombres \|x| et \|y| si \|x| est plus grand que \|y| et renvoie une
  erreur sinon.

\item \|NombreA2Chiffres plus(NombreA2Chiffres x, NombreA2Chiffres y)|
  renvoie le nombre obtenu en faisant le produit des deux nombres
  \|x| et \|y|.

\end{itemize}
Ces trois dernières opérations ignorent le chiffre des centaines.

Les valeurs de types \|Chiffre| et \|NombreA2Chiffres| peuvent être
affichées avec les fonctions \|println| et \|echo| habituelles sous
Javascool. 

Nous fournissons de plus une fonction
\begin{lstlisting}
  String nombreVersChaine(Chiffre[] nombre)
\end{lstlisting}
qui permet d'afficher un nombre représenté par un tableau de chiffre.

\begin{exemple}
\|echo(plus(nb(UN,TROIS),nb(DEUX,QUATRE)));| provoque l'affichage
\|37|
\end{exemple}

\begin{exemple}
  Le tableau \|nombre| suivant peut être affiché avec les
  instructions:
\begin{lstlisting}
Chiffre[] nombre = { ZERO, TROIS, HUIT };
System.out.println(nombreVersChaine(nombre));
\end{lstlisting}
pour produire l'affichage \|038|.
\end{exemple}

\begin{question}
  Écrire une fonction
\begin{lstlisting}
Chiffre[] addition(int taille, Chiffre[] n1, Chiffre[] n2)
\end{lstlisting}
qui construit le tableau de chiffre résultant de l'addition des
nombres \|n1| et \|n2|, ces nombres étant représentés par des tableaux
de taille \|taille|. Le résultat de l'addition devra être exprimé dans
un tableau de même taille et une erreur devra être lancée si cela n'est
pas possible.

Pour réaliser un test et lancer une erreur en cas d'échec, javascool
propose une fonction particulière
\begin{lstlisting}
void assertion(boolean test, String message) 
\end{lstlisting}
Si l'argument \|test| est vrai, l'exécution continue sans afficher de
message particulier, sinon toute l'exécution courante se termine avec
le message d'erreur proposé en argument.
\end{question}

\begin{question}
  Écrire une fonction
\begin{lstlisting}
Chiffre[] multiplication(int taille, Chiffre[] n1, Chiffre[] n2)
\end{lstlisting}
qui calcule le tableau de chiffres résultant de la multiplication des
nombres \|n1| et \|n2|. Ces nombres sont représentés par des tableaux
de taille \|taille|. Le résultat de la multiplication devra être
exprimé dans un tableau de même taille et une erreur devra être lancée
si cela n'est pas possible.
\end{question}

\begin{question}
  Écrire une fonction
\begin{lstlisting}
Chiffre[] soustraction(int taille, Chiffre[] n1, Chiffre[] n2)
\end{lstlisting}
  qui calcule le tableau de chiffre résultant de la soustraction des
  nombres \|n1| et \|n2|. Ces nombres sont représentés par des
  tableaux de taille \|taille|. Le résultat de la soustraction devra
  être exprimé dans un tableau de même taille et une erreur devra être
  lancée si \|n1| n'est pas supérieur ou égal à \|n2|.

  Pour cette fonction, il sera utile d'utiliser la fonction
\begin{lstlisting}
boolean estPlusPetit(NombreA2Chiffres x, NombreA2Chiffres y) 
\end{lstlisting}
pour tester si un nombre à deux chiffres \|x| est inférieur ou égal à un autre
nombre à deux chiffres \|y|.
\end{question}

\begin{question}[Pour les programmeurs Java]
  Reprendre les questions précédentes en manipulant cette fois des
  tableaux de bits. Un bit pourra être représenté par un simple
  booléen.
\end{question}

\end{activite}

\end{document}

% LocalWords:  IA-IPR David Pichardie Luc Javascool Télécharger equal peut-on
% LocalWords:  readString copier-coller marion chaineVersLettres abeu void int
% LocalWords:  lettresVersChaine indexMin t.length triSelection boolean enum nb
% LocalWords:  testAnagramme ZERO NombreA encapsule unite println echo
% LocalWords:  nombreVersChaine
