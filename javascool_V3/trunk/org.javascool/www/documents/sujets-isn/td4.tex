\documentclass[a4paper,11pt]{article}

%% \LoadClass[a4paper,11pt]{article}

\usepackage{a4wide,graphicx}%

%\usepackage[francais]{babel}%
\usepackage[utf8]{inputenc}%
\usepackage[T1]{fontenc}%

\usepackage{url} \urlstyle{sf}%

\usepackage{mathpazo}%
\let\bfseriesaux=\bfseries%
\renewcommand{\bfseries}{\sffamily\bfseriesaux}

\newcounter{probleme} \newcounter{question}[probleme]
\setcounter{probleme}{0} \def\theprobleme{\Roman{probleme}}
\def\thequestion{\theprobleme.\arabic{question}}

\newenvironment{probleme}[1]%
{\refstepcounter{probleme}
\subsection*{Activité \theprobleme. #1}}%
{}

\newenvironment{question}%
{\refstepcounter{question} \description%
\item[Exercice \thequestion.] \bgroup}%
{\egroup\enddescription}

\newenvironment{remarque}%
{\description\item[Remarque.]\sl}%
{\enddescription}

\newenvironment{exemple}%
{\description\item[Exemple.]\sl}%
{\enddescription}

\newenvironment{programme}%
{\verbatim}{\endverbatim}%

\usepackage{listings}%

\lstset{%
  basicstyle=\sffamily,%
  columns=fullflexible,%
   language=java,%
   frame=lb,%
   frameround=fftf,%
}%

\def\|{\lstinline|}%

\lstnewenvironment{lstlistingtt}%
{\lstset{%
    basicstyle=\tt,%
    columns=flexible,%
  }}%
{}

\lstnewenvironment{lstlistingsh}%
{\lstset{%
  language=sh,%
  }}%
{}

%% \lstMakeShortInline{|}

\parskip=0.3\baselineskip%
\sloppy%

\begin{document}

\title{Enseignement de spécialité\\
Informatique et sciences du numérique\\
Formation des IA-IPR et chargés de mission\\
Atelier de programmation 5}

\date{Jeudi 17 mars}

\author{David Pichardie, Luc Bougé}
\maketitle

Dans cette activité, nous allons manipuler un réseau de villes françaises 
reliées par des routes pour effectuer des calculs d'itinéraires. 

Prendre connaissance des fonctionnalités de la \emph{proglet}
Javascool "Jouer avec une carte de France" (onglet "Document de la
proglet").

\begin{question}
Ecrire une fonction
\begin{lstlisting}
void afficheVille(String ville) 
\end{lstlisting}
qui affiche la ville de nom \|ville| sur la carte. La fonction doit
lancer une erreur (avec la fonction Javascool \|assertion|) si cette
ville n'appartient pas à la base de donnée.

Proposer une variante
\begin{lstlisting}
void afficheVilleAvecNumero(String ville, int num) 
\end{lstlisting}
qui affiche un numéro \|num| au dessus de cette ville.
\end{question}

\begin{question}
Utiliser les fonctions précédentes pour afficher  toutes les villes de la base de donnée.
\begin{lstlisting}
void afficheToutesVilles() 
\end{lstlisting}
\end{question}

\begin{question}
Ecrire une fonction  
\begin{lstlisting}
void afficheRouteDirecte(String ville1, String ville2, IntensiteRoute intensite) 
\end{lstlisting}
qui relie avec un segment d'intensité \|intensite| les deux villes de
noms \|ville1| et \|ville2|. Là encore, la fonction est suceptible de
lancer une erreur.
\end{question}

\begin{question}
Utiliser la fonction précédente pour afficher  toutes les routes de la base de donnée.
\begin{lstlisting}
void afficheToutesRoutesDirectes() {
\end{lstlisting}
\end{question}

\begin{question}
Ecrire une fonction
\begin{lstlisting}
void afficherChemin(List<String> chemin) 
\end{lstlisting}
qui affiche sur la carte le chemin représenté par la séquence de noms
de ville \|chemin|. 

Proposer une variante
\begin{lstlisting}
void afficherCheminAvecNumeros(List<String> chemin) 
\end{lstlisting}
qui affiche aussi l'ordre de passage dans chaque ville.

Utiliser cette fonction avec la fonction fournie dans Javascool
\begin{lstlisting}
List<String> plusCourtCheminGogleMap(String depart, String arrivee)   
\end{lstlisting}
qui calcule un chemin sous forme d'une liste de noms de ville afin de relier
la ville de nom \|depart| à celle de nom \|arrivee| en suivant uniquement
des routes existantes dans la base de donnée.
\end{question}

\begin{question}
Ecrire une fonction
\begin{lstlisting}
int distance(String ville1, String ville2) 
\end{lstlisting}
qui calcule la distance en km entre deux ville reliées par une route
directe. Si une route directe n'existe pas, la fonction doit renvoyer
le plus grand entier du type \|int| (nommé \|maxInteger|).
\end{question}

\begin{question}
Ecrire une fonction
\begin{lstlisting}
int longueurChemin(List<String> chemin) 
\end{lstlisting}  
qui calcule la longueur totale d'un chemin représenté par la séquence
de noms de ville \|chemin|. Que se passe-t-il si ce chemin ne passe
pas par une route directe ?
\end{question}


% \begin{lstlisting}
% String PlusProche(List<String> groupe, Map<String,Integer> distMap)
% \end{lstlisting}

% \begin{lstlisting}
% void MiseAjourDistance(String ville0, Map<String,Integer> distMap,  Map<String,String> pred) 
% \end{lstlisting}

% \begin{lstlisting}
% List<String> plusCourtChemin(String depart, String arrivee) 
% \end{lstlisting}

%\url{http://commons.wikimedia.org/wiki/Image:France_cities.pdf?uselang=fr}


\end{document}
